\documentclass[10pt,ignorenonframetext,,aspectratio=149]{beamer}
\setbeamertemplate{caption}[numbered]
\setbeamertemplate{caption label separator}{: }
\setbeamercolor{caption name}{fg=normal text.fg}
\usepackage{lmodern}
\usepackage{amssymb,amsmath}
\usepackage{ifxetex,ifluatex}
\usepackage{fixltx2e} % provides \textsubscript
\ifnum 0\ifxetex 1\fi\ifluatex 1\fi=0 % if pdftex
  \usepackage[T1]{fontenc}
  \usepackage[utf8]{inputenc}
\else % if luatex or xelatex
  \ifxetex
    \usepackage{mathspec}
  \else
    \usepackage{fontspec}
  \fi
  \defaultfontfeatures{Ligatures=TeX,Scale=MatchLowercase}
  \newcommand{\euro}{€}
\fi
% use upquote if available, for straight quotes in verbatim environments
\IfFileExists{upquote.sty}{\usepackage{upquote}}{}
% use microtype if available
\IfFileExists{microtype.sty}{%
\usepackage{microtype}
\UseMicrotypeSet[protrusion]{basicmath} % disable protrusion for tt fonts
}{}

% Comment these out if you don't want a slide with just the
% part/section/subsection/subsubsection title:
\AtBeginPart{
  \let\insertpartnumber\relax
  \let\partname\relax
  \frame{\partpage}
}
\AtBeginSection{
  \let\insertsectionnumber\relax
  \let\sectionname\relax
  \frame{\sectionpage}
}
\AtBeginSubsection{
  \let\insertsubsectionnumber\relax
  \let\subsectionname\relax
  \frame{\subsectionpage}
}

\setlength{\emergencystretch}{3em}  % prevent overfull lines
\providecommand{\tightlist}{%
  \setlength{\itemsep}{0pt}\setlength{\parskip}{0pt}}
\setcounter{secnumdepth}{0}

\title{Chapter 2 - An Introduction to Forwards and Options}
\subtitle{Finance 6470: Derivatives Markets}
\author{Tyler J. Brough}
\date{}

%% Here's everything I added.
%%--------------------------

\usepackage{graphicx}
\usepackage{rotating}
%\setbeamertemplate{caption}[numbered]
\usepackage{hyperref}
\usepackage{caption}
\usepackage[normalem]{ulem}
%\mode<presentation>
\usepackage{wasysym}
%\usepackage{amsmath}


% Get rid of navigation symbols.
%-------------------------------
\setbeamertemplate{navigation symbols}{}

% Optional institute tags and titlegraphic.
% Do feel free to change the titlegraphic if you don't want it as a Markdown field.
%----------------------------------------------------------------------------------
\institute{Department of Finance and Economics}

% \titlegraphic{\includegraphics[width=0.3\paperwidth]{\string~/Dropbox/teaching/clemson-academic.png}} % <-- if you want to know what this looks like without it as a Markdown field. 
% -----------------------------------------------------------------------------------------------------
\titlegraphic{\includegraphics[width=0.3\paperwidth]{\string~./images/vertical-logo-blue.png}}

% Some additional title page adjustments.
%----------------------------------------
\setbeamertemplate{title page}[empty]
%\date{}
\setbeamerfont{subtitle}{size=\small}

\setbeamercovered{transparent}

% Some optional colors. Change or add as you see fit.
%---------------------------------------------------
\definecolor{clemsonpurple}{HTML}{522D80}
% \definecolor{clemsonorange}{HTML}{EA6A20}
 \definecolor{clemsonorange}{HTML}{F66733}
\definecolor{uiucblue}{HTML}{003C7D}
\definecolor{uiucorange}{HTML}{F47F24}
%\definecolor{usublue}{HTML}{2D3280}
%\definecolor{usublue}{HTML}{000066}
%\definecolor{usublue}{HTML}{273849}
%\definecolor{usublue}{HTML}{78281F}
%\definecolor{usublue}{HTML}{A04000}


%\definecolor{usublue}{HTML}{3498DB}
%\definecolor{bluegray}{HTML}{334760}
\definecolor{usublue}{HTML}{C0392B}
\definecolor{bluegray}{HTML}{34495E}

\definecolor{gray}{HTML}{446280}

% Some optional color adjustments to Beamer. Change as you see fit.
%------------------------------------------------------------------
%\setbeamercolor{frametitle}{fg=clemsonpurple,bg=white}
\setbeamercolor{frametitle}{fg=usublue,bg=white}
\setbeamercolor{title}{fg=usublue,bg=white}
\setbeamercolor{local structure}{fg=usublue}
\setbeamercolor{section in toc}{fg=usublue,bg=white}
% \setbeamercolor{subsection in toc}{fg=clemsonorange,bg=white}
\setbeamercolor{footline}{fg=usublue!50, bg=white}
\setbeamercolor{block title}{fg=gray,bg=white}


\let\Tiny=\tiny


% Sections and subsections should not get their own damn slide.
%--------------------------------------------------------------
\AtBeginPart{}
\AtBeginSection{}
\AtBeginSubsection{}
\AtBeginSubsubsection{}

% Suppress some of Markdown's weird default vertical spacing.
%------------------------------------------------------------
\setlength{\emergencystretch}{0em}  % prevent overfull lines
\setlength{\parskip}{0pt}


% Allow for those simple two-tone footlines I like. 
% Edit the colors as you see fit.
%--------------------------------------------------
\defbeamertemplate*{footline}{my footline}{%
    \ifnum\insertpagenumber=1
    \hbox{%
        \begin{beamercolorbox}[wd=\paperwidth,ht=.8ex,dp=1ex,center]{}%
      % empty environment to raise height
        \end{beamercolorbox}%
    }%
    \vskip0pt%
    \else%
        \Tiny{%
            \hfill%
		\vspace*{1pt}%
            \insertframenumber/\inserttotalframenumber \hspace*{0.1cm}%
            \newline%
            \color{usublue}{\rule{\paperwidth}{0.4mm}}\newline%
            \color{bluegray}{\rule{\paperwidth}{.4mm}}%
        }%
    \fi%
}

% Various cosmetic things, though I must confess I forget what exactly these do and why I included them.
%-------------------------------------------------------------------------------------------------------
\setbeamercolor{structure}{fg=blue}
\setbeamercolor{local structure}{parent=structure}
\setbeamercolor{item projected}{parent=item,use=item,fg=usublue,bg=white}
\setbeamercolor{enumerate item}{parent=item}

% Adjust some item elements. More cosmetic things.
%-------------------------------------------------
\setbeamertemplate{itemize item}{\color{usublue}$\bullet$}
\setbeamertemplate{itemize subitem}{\color{usublue}\scriptsize{$\bullet$}}
\setbeamertemplate{itemize/enumerate body end}{\vspace{.6\baselineskip}} % So I'm less inclined to use \medskip and \bigskip in Markdown.

% Automatically center images
% ---------------------------
% Note: this is for ![](image.png) images
% Use "fig.align = "center" for R chunks

\usepackage{etoolbox}

\AtBeginDocument{%
  \letcs\oig{@orig\string\includegraphics}%
  \renewcommand<>\includegraphics[2][]{%
    \only#3{%
      {\centering\oig[{#1}]{#2}\par}%
    }%
  }%
}

% I think I've moved to xelatex now. Here's some stuff for that.
% --------------------------------------------------------------
% I could customize/generalize this more but the truth is it works for my circumstances.

\ifxetex
\setbeamerfont{title}{family=\fontspec{serif}}
\setbeamerfont{frametitle}{family=\fontspec{serif}}
\usepackage[font=small,skip=0pt]{caption}
 \else
 \fi

% Okay, and begin the actual document...

\begin{document}
\frame{\titlepage}

\begin{frame}{Introduction}
\protect\hypertarget{introduction}{}
\begin{itemize}
\tightlist
\item
  Basic derivatives contracts

  \begin{itemize}
  \tightlist
  \item
    Forward contracts (long \& short)
  \item
    Call options
  \item
    Put options
  \end{itemize}
\item
  Types of positions

  \begin{itemize}
  \tightlist
  \item
    Long position (buyer)
  \item
    Short position (seller)
  \end{itemize}
\item
  Graphical representation

  \begin{itemize}
  \tightlist
  \item
    Payoff diagrams (does not take into account upfront costs)
  \item
    Profit diagrams
  \end{itemize}
\end{itemize}
\end{frame}

\begin{frame}{Forward Contracts}
\protect\hypertarget{forward-contracts}{}
\begin{itemize}
\item
  Definition: a binding agreement (obligation) to buy or sell an
  underlying asset in the future, at a price set today
\item
  Futures contracts are the same as forwards in principle except for
  some institutional and pricing differences

  \begin{itemize}
  \tightlist
  \item
    (futures are traded on an exchange and are marked-to-market daily)
  \end{itemize}
\item
  A forward contract specifies

  \begin{itemize}
  \tightlist
  \item
    The features and quantity of the asset to be delivered
  \item
    The delivery logistics, such as time, date, and place
  \item
    The price the buyer will pay at the time of delivery
  \end{itemize}
\end{itemize}
\end{frame}

\begin{frame}{The Payoff on a Forward Contract}
\protect\hypertarget{the-payoff-on-a-forward-contract}{}
\begin{itemize}
\item
  Payoff for a contract is its value at expiration
\item
  Payoff for

  \begin{itemize}
  \tightlist
  \item
    Long forward: \(S_{T} - F_{0,T}\)
  \item
    Short forward: \(F_{0,T} - S_{T}\)
  \end{itemize}
\item
  Example 2.1 S\&R (special and rich) index
\end{itemize}
\end{frame}

\begin{frame}{Payoff Diagrams for Forwards}
\protect\hypertarget{payoff-diagrams-for-forwards}{}
\end{frame}

\begin{frame}{Forward Versus Outright Purchase}
\protect\hypertarget{forward-versus-outright-purchase}{}
\vspace{50mm}

\[
\begin{aligned}
\mbox{Forward} + \mbox{bond} &= \mbox{Spot price at expiration} - \$1,020 + \$1,020 \\
                             &= \mbox{Spot price at expiration}
\end{aligned}
\]
\end{frame}

\begin{frame}{Additional Considerations}
\protect\hypertarget{additional-considerations}{}
\begin{itemize}
\tightlist
\item
  Type of settlement

  \begin{itemize}
  \tightlist
  \item
    Cash settlement: less costly and more practical
  \item
    Physical delivery: often avoided due to significant costs
  \end{itemize}
\item
  Credit risk of the counter pary

  \begin{itemize}
  \tightlist
  \item
    Major issue for over-the-counter contracts

    \begin{itemize}
    \tightlist
    \item
      Credit check, collateral, bank letter of credit
    \end{itemize}
  \item
    Less severe for exchange-traded contracts

    \begin{itemize}
    \tightlist
    \item
      Exchange guarantees transactions, requires collateral
    \end{itemize}
  \end{itemize}
\end{itemize}
\end{frame}

\begin{frame}{Call Options}
\protect\hypertarget{call-options}{}
\begin{itemize}
\item
  A non-binding agreement (right but not an obligation) to buy an asset
  in the future, at a price set today
\item
  Preserves the upside potential, while at the same time eliminating the
  unpleasant downside (for the buyer)
\item
  The seller of a call option is obligated to deliver if asked (the long
  position of the option holds the optionality)
\end{itemize}
\end{frame}

\begin{frame}{Examples}
\protect\hypertarget{examples}{}
\end{frame}

\begin{frame}{Definition and Terminology}
\protect\hypertarget{definition-and-terminology}{}
\begin{itemize}
\item
  A call option gives the owner the right but not the obligation to buy
  the underlying asset at a predetermined price during a predetermined
  time period
\item
  Strike (or exercise) price: the amount paid by the option buyer for
  the asset if she decides to exercise
\item
  Exercise: the act of paying the strike price to buy the asset
\item
  Expiration (expiry): the date by which the option must be exercised

  \begin{itemize}
  \tightlist
  \item
    European: can be exercised only at expiration date
  \item
    American: can be exercised at any time before expiration
  \item
    Burmudan: can be exercised during specified periods
  \end{itemize}
\end{itemize}
\end{frame}

\begin{frame}{Payoff/Profit of a Long Call Position (Purchased)}
\protect\hypertarget{payoffprofit-of-a-long-call-position-purchased}{}
\begin{itemize}
\item
  \(\mbox{Long call payoff} = \max{\{0, S_{T} - K\}}\)
\item
  Profit = Payoff - future value of option premium
\item
  Examples
\end{itemize}

\vspace{40mm}
\end{frame}

\begin{frame}{Diagrams for Purchased Call}
\protect\hypertarget{diagrams-for-purchased-call}{}
\end{frame}

\begin{frame}{Payoff \& Profit of a Short Call Position (Written)}
\protect\hypertarget{payoff-profit-of-a-short-call-position-written}{}
\begin{itemize}
\item
  \(\mbox{Short call payoff} = -\max{\{0, S_{T} - K\}}\)
\item
  Profit = Payoff + future value of option premium
\item
  Example 2.7
\end{itemize}

\vspace{40mm}
\end{frame}

\begin{frame}{Put Options}
\protect\hypertarget{put-options}{}
\begin{itemize}
\item
  A put option gives the owner the right but now the obligation to sell
  the underlying asset at a predetermined price during a predetermined
  time period
\item
  The seller of a put option is obligated to buy if asked
\item
  Payoff/profit of a long put (purchased)

  \begin{itemize}
  \tightlist
  \item
    \(\mbox{Long put payoff} = \max{\{0, K - S_{T}\}}\)
  \item
    Profit = Payoff - future value of option premium
  \end{itemize}
\item
  Payoff/profit of a short put (written)

  \begin{itemize}
  \tightlist
  \item
    \(\mbox{Short put payoff} = -\max{\{0, K - S_{T}\}}\)
  \item
    Profit = Payoff + future value of option premium
  \end{itemize}
\end{itemize}
\end{frame}

\begin{frame}{Put Option Examples}
\protect\hypertarget{put-option-examples}{}
\end{frame}

\begin{frame}{Profit for a Long Put Position}
\protect\hypertarget{profit-for-a-long-put-position}{}
\end{frame}

\begin{frame}{A Few Items to Note}
\protect\hypertarget{a-few-items-to-note}{}
\begin{itemize}
\item
  A call option becomes more profitable when the underlying asset
  appreciates in value
\item
  A put option becomes more profitable when the underlying asset
  depreciates in value
\item
  Moneyness:

  \begin{itemize}
  \tightlist
  \item
    In-the-money (ITM): positive payoff if exercised immediately
  \item
    At-the-money (ATM): zero payoff if exercised immediately
  \item
    Out-of-the-money: (OTM): negative payoff if exercised immediately
  \end{itemize}
\end{itemize}
\end{frame}

\begin{frame}{Option and Forward Positions: A Summary}
\protect\hypertarget{option-and-forward-positions-a-summary}{}
\end{frame}


%\section[]{}
%\frame{\small \frametitle{Table of Contents}
%\tableofcontents}
\end{document}
